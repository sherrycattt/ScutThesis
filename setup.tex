% !TeX root = ./main.tex

% 论文基本信息配置

\thusetup{
  %******************************
  % 注意:
  %   1. 配置里面不要出现空行
  %   2. 不需要的配置信息可以删除
  %   3. 建议先阅读文档中所有关于选项的说明
  %******************************
  %
  % 输出格式
  %   选择打印版(print)或用于提交的电子版(electronic),前者会插入空白页以便直接双面打印
  %
  output = electronic,
  %
  % 标题
  %   可使用“\\”命令手动控制换行
  %
  title  = {低光照条件下的图像增强和识别\\关键技术研究},
  title* = {Research on Key Technologies of Image  Enhancement  and Recognition in Low-Light Conditions},
  % 
  institute = {华南理工大学},
  institute* = {South China University of Technology},
  %
  % 学位
  %   1. 学术型
  %      - 中文
  %        需注明所属的学科门类,例如:
  %        哲学、经济学、法学、教育学、文学、历史学、理学、工学、农学、医学、
  %        军事学、管理学、艺术学
  %      - 英文
  %        博士:Doctor of Philosophy
  %        硕士:
  %          哲学、文学、历史学、法学、教育学、艺术学门类,公共管理学科
  %          填写“Master of Arts“,其它填写“Master of Science”
  %   2. 专业型
  %      直接填写专业学位的名称,例如:
  %      教育博士、工程硕士等
  %      Doctor of Education, Master of Engineering
  %   3. 本科生不需要填写
  %
  degree-name  = {工学博士},
  degree-name* = {Doctor of Philosophy},
  %
  % 培养单位
  %   填写所属院系的全名
  %
  department = {计算机科学与工程学院},
  %
  % 学科
  %   1. 学术型学位
  %      获得一级学科授权的学科填写一级学科名称,其他填写二级学科名称
  %   2. 工程硕士
  %      工程领域名称
  %   3. 其他专业型学位
  %      不填写此项
  %   4. 本科生填写专业名称,第二学位论文需标注“(第二学位)”
  %
  discipline  = {计算机科学与技术},
  discipline* = {Computer Science and Technology},
  %
  % 姓名
  %
  author  = {梁锦秀},
  author* = {Liang Jinxiu},
  %
  % 指导教师
  %   中文姓名和职称之间以英文逗号“,”分开,下同
  %
  supervisor  = {许勇, 教授},
  supervisor* = {Professor Xu Yong},
  %
  % 副指导教师
  %
  %associate-supervisor  = {陈文光, 教授},
  %associate-supervisor* = {Professor Chen Wenguang},
  %
  % 联合指导教师
  %
  % co-supervisor  = {某某某, 教授},
  % co-supervisor* = {Professor Mou Moumou},
  %
  % 日期
  %   使用 ISO 格式;默认为当前时间
  %
   date = {2021-04-12},
  %
  % 是否在中文封面后的空白页生成书脊(默认 false)
  %
  include-spine = false,  
  %   
  % 是否使用pdf封面(默认 \@empty
  %
  include-cover = true,
  % cover-file = {figures/cover.pdf},
  %
  % 密级和年限
  %   秘密, 机密, 绝密
  %
  % secret-level = {秘密},
  % secret-year  = {10},
  %
  % 博士后专有部分
  %
  % clc                = {分类号},
  % udc                = {UDC},
  % id                 = {编号},
  % discipline-level-1 = {计算机科学与技术},  % 流动站(一级学科)名称
  % discipline-level-2 = {系统结构},          % 专业(二级学科)名称
  % start-date         = {2011-07-01},        % 研究工作起始时间
  toc-chapter-style = times,
  number-separator = {-},
  student-id = {xxxxxxx},
}

%---------------------------------------------------------------------------%
%->> Load packages 载入所需的宏包
%---------------------------------------------------------------------------%

% 定理类环境宏包
\usepackage{amsthm}
% 也可以使用 ntheorem
% \usepackage[amsmath,thmmarks,hyperref]{ntheorem}

\thusetup{
  %
  % 数学字体
  math-style = GB,  % GB | ISO | TeX
  math-font  = xits,  % sitx | xits | libertinus
}

% 可以使用 nomencl 生成符号和缩略语说明
% \usepackage{nomencl}
% \makenomenclature

% 表格加脚注
\usepackage{threeparttable}

% 表格中支持跨行
\usepackage{multirow}

% 固定宽度的表格。
% \usepackage{tabularx}

% 跨页表格
\usepackage{longtable}

% The package allows rows and columns to be coloured, and even individual cells.
\usepackage{colortbl}

% 量和单位
\usepackage{siunitx}

% 参考文献使用 BibTeX + natbib 宏包
% 顺序编码制
\usepackage[sort]{natbib}
\bibliographystyle{thuthesis-numeric}

% 参考文献中使用\citet{}时的引用方式
% \def\bibetal{等}
% \def\biband{和}

% The package subfigure is now considered obsolete: it was superseded by subfig, but users may find the more recent subcaption package more satisfactory.
%\usepackage{subfigure}
%\usepackage{subfig}

% Draw dash-lines in array/tabular. 
\usepackage{arydshln}

% Marking things to do in a LaTeX document.
\usepackage[backgroundcolor=yellow]{todonotes}

% allow the placement of graphics relative to the “current position” using additional optional arguments of \includegraphics.
\usepackage{graphbox}
\usepackage{adjustbox}

%  provides an easy-to-use interface to the bbding symbol set
\usepackage{bbding}

% create algorithms
\usepackage{algorithm}
\usepackage[noend]{algpseudocode}

% hyperref 宏包在最后调用
\usepackage{hyperref}


%---------------------------------------------------------------------------%
%->> Configuration command
%---------------------------------------------------------------------------%

%-> Extensions and directories for graphics
% 定义所有的图片文件在 figures 子目录下
\graphicspath{{figures/}}

%- Declare graphic extensions for automatic selection when including graphics
%- via avoiding supplying graphic extensions in \includegraphics command,
%- the source file can be more general and adaptive
\DeclareGraphicsExtensions{.pdf,.png,.jpg,.eps,.tif,.bmp,.gif}%

% 数学命令
\makeatletter
\newcommand\dif{%  % 微分符号
  \mathop{}\!%
  \ifthu@math@style@TeX
    d%
  \else
    \mathrm{d}%
  \fi
}
\makeatother

% alignment in table
\renewcommand{\bfseries}{\fontseries{b}\selectfont}
\robustify\bfseries
\newrobustcmd{\BF}{\bfseries}

%-> Layout, space, and style
%\linespread{1.5}% 1.5 for "one and a half" line spacing, and 2.0 for "double" line spacing
%\setlength{\parskip}{0.5ex plus 0.25ex minus 0.25ex}% skip space a paragraph
% \setcounter{secnumdepth}{4}% depth for section numbering, default is 2(subsub)
% \setcounter{tocdepth}{2}% depth for the table of contents

\newcommand{\hiddensubsection}[1]{
	\stepcounter{subsection}
	\subsection*{{\rmfamily \arabic{chapter}.\arabic{section}.\arabic{subsection}}\hspace{1em}{#1}}
}

%---------------------------------------------------------------------------%
%->> User defined commands
%---------------------------------------------------------------------------%

%%Word abbreviations
% \newcommand{\st}{\text{s.t.}\ \ }
\newcommand\eg{\emph{e.g}~} 
\newcommand\Eg{\emph{E.g}~}
\newcommand\ie{\emph{i.e}~} 
\newcommand\Ie{\emph{I.e}~}
\newcommand\cf{\emph{c.f}~} 
\newcommand\Cf{\emph{C.f}~}
\newcommand\etc{\emph{etc}~} 
\newcommand\vsu{\emph{vs}~}
\newcommand\wrt{w.r.t~} 
\newcommand\dof{d.o.f~}
\newcommand\etal{\emph{et al}~}

%argmin and argmax
\DeclareMathOperator*{\argmin}{argmin}
\DeclareMathOperator*{\argmax}{argmax}


% Comments with highlights
\newcommand{\red}[1]{\textcolor[rgb]{1.00, 0.00, 0.00}{{#1}}} % comments in red
\newcommand{\green}[1]{\textcolor[rgb]{0.00, 1.00, 0.00}{{#1}}} % comments in green
\newcommand{\blue}[1]{\textcolor[rgb]{0.00, 0.00, 1.00}{{#1}}} % comments in blue
\newcommand{\black}[1]{\textcolor[rgb]{0.00, 0.00, 0.00}{{#1}}} % comments in black


%-> Math functions
%-
%- International standard layout rules (from isomath package)
%- The overall rule is that symbols representing math quantities or variables should
%- be italicised, symbols representing units or labels are unitalicised (roman).
%- Symbols for vectors and matrices are bold italic, symbols for tensors are 
%- sans-serif bold italic.
%- The above rules apply equally to letter symbols from the Greek and 
%- the Latin alphabet.
%- More information may be found in <<The LaTeX Mathematics Companion>>
%- However, math typefaces vary from field to field. To keep consistent typography
%- and easy adaption, it it always best to create a corresponding command for 
%- variables in each math category.  

%%Variable style
\newcommand{\vect}[1]{{\boldsymbol{#1}}} %%Vector in bold italic
\newcommand{\matx}[1]{{\boldsymbol{#1}}} %%Matrix in bold italic
\newcommand{\unitmt}[1]{\boldsymbol{\mathbf{#1}}} %%Identity matrix in bold roman
\newcommand{\tens}[1]{\boldsymbol{\mathsf{#1}}} %%Tensor in sans-serif bold italic
\newcommand{\unitts}[1]{\boldsymbol{\mathsf{#1}}}%%Identity tensor in sans-serif bold
\newcommand{\set}[1]{{\mathbb{#1}}}			%set notation
\newcommand{\opt}[1]{{\mathcal{#1}}}		%operator noation
\providecommand{\unit}[1]{\,\mathrm{#1}}% units in roman
\providecommand{\const}[1]{\mathrm{#1}}% math constants, functions
\providecommand{\desc}[1]{\mathrm{#1}}% descriptive superscripts and subscripts in roman type 
\providecommand{\div}{\operatorname{div}}% divergence operator
\providecommand{\order}{\operatorname{O}}% order operator
\providecommand{\trace}{\operatorname{tr}}% trace operator
\providecommand{\diag}{\operatorname{diag}}% diagonal
\providecommand{\def}{\operatorname{def}}% define
