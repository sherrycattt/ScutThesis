% !TeX root = ../main.tex

% \begin{resolution}[name={答辩委员签名的答辩决议书}]
%     \noindent \textbf{Ⅳ - 2答辩委员会对论文的评定意见}


%     \begin{table}
%         \begin{longtable}{|m{1.20cm}|m{13.80cm}|}
%             \hline
%             \multicolumn{2}{|m{15.00cm}|}{
%                 论文提出了……

%                 论文取得的主要创新性成果包括:

%                 1. ……

%                 2. ……

%                 3. ……

%                 论文工作表明作者在×××××具有×××××知识,具有××××能力,论文××××,答辩××××。

%                 答辩委员会表决,(×票/一致)同意通过论文答辩,并建议授予×××(姓名)×××(门类)学博士学位。
%             } \tabularnewline[10.00cm] \hline 
%             \multicolumn{2}{|m{15.00cm}|}{\begin{tabular}{l}
%                 论文答辩日期:\underbar{\ \ \ \ \ \ \ \ \ \ \ \ \ \ \ \ }年\underbar{\ \ \ \ \ \ \ \ }月\underbar{\ \ \ \ \ \ \ \ }日 \\
%                 答辩委员会委员共\_\_\_\_\_\_\_人,到会委员\_\_\_\_\_\_\_人 \\
%                  表决票数:优秀(  )票;良好(  )票;及格(  )票;不及格(  )票 \\
%                   表决结果(打``$\mathrm{\sqrt{}}$''):优秀(  );良好(  );及格(  );不及格(  )   \\ 
%                决议:同意授予博士学位(  )   不同意授予博士学位(  )
%             \end{tabular}} \tabularnewline[2.00cm] \hline 
%             答辩委员会成员签名 & 
%             \begin{tabular}{b{4.60cm}b{4.60cm}b{4.60cm}}
%                 & & \\
%                 \underbar{\ \ \ \ \ \ \ \ \ \ \ \ \ \ \ \ \ \ \ \ (主席)}  &
%                 \underbar{\ \ \ \ \ \ \ \ \ \ \ \ \ \ \ \ \ \ \ \ \ \ \ \ \ \ \ \ \ \ \ \ \ \ \ \ }   &
%                 \underbar{\ \ \ \ \ \ \ \ \ \ \ \ \ \ \ \ \ \ \ \ \ \ \ \ \ \ \ \ \ \ \ \ \ \ \ \ }  \\
%                 & & \\
%                 \underbar{\ \ \ \ \ \ \ \ \ \ \ \ \ \ \ \ \ \ \ \ \ \ \ \ \ \ \ \ \ \ \ \ \ \ \ \ }   & 
%                 \underbar{\ \ \ \ \ \ \ \ \ \ \ \ \ \ \ \ \ \ \ \ \ \ \ \ \ \ \ \ \ \ \ \ \ \ \ \ }  &
%                 \underbar{\ \ \ \ \ \ \ \ \ \ \ \ \ \ \ \ \ \ \ \ \ \ \ \ \ \ \ \ \ \ \ \ \ \ \ \ }  \\
%             \end{tabular} \tabularnewline \hline
%         \end{longtable}
%     \end{table}

% \end{resolution}





% 也可以导入 Word 版转的 PDF 文件
\begin{committee}[file=figures/resolution.pdf]
\end{committee}
