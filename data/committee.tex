% !TeX root = ../main.tex

\begin{committee}[name={学位论文指导小组、公开评阅人和答辩委员会名单}]

  \newcolumntype{C}[1]{@{}>{\centering\arraybackslash}p{#1}}

  \section*{指导小组名单}

  \begin{center}
    \begin{tabular}{C{3cm}C{3cm}C{9cm}@{}}
      李XX & 教授     & 清华大学 \\
      王XX & 副教授   & 清华大学 \\
      张XX & 助理教授 & 清华大学 \\
    \end{tabular}
  \end{center}


  \section*{公开评阅人名单}

  \begin{center}
    \begin{tabular}{C{3cm}C{3cm}C{9cm}@{}}
      刘XX & 教授   & 清华大学                    \\
      陈XX & 副教授 & XXXX大学                    \\
      杨XX & 研究员 & 中国XXXX科学院XXXXXXX研究所 \\
    \end{tabular}
  \end{center}


  \section*{答辩委员会名单}

  \begin{center}
    \begin{tabular}{C{2.75cm}C{2.98cm}C{4.63cm}C{4.63cm}@{}}
      主席 & 赵XX                  & 教授                    & 清华大学       \\
      委员 & 刘XX                  & 教授                    & 清华大学       \\
          & \multirow{2}{*}{杨XX} & \multirow{2}{*}{研究员} & 中国XXXX科学院 \\
          &                       &                         & XXXXXXX研究所  \\
          & 黄XX                  & 教授                    & XXXX大学       \\
          & 周XX                  & 副教授                  & XXXX大学       \\
      秘书 & 吴XX                  & 助理研究员              & 清华大学       \\
    \end{tabular}
  \end{center}

  
  % \clearpage
  % ~\\[1.5cm]
  % \begin{figure}\centering
  %   \includegraphics*[width=4.76in, height=1.08in, keepaspectratio=false]{scut_logo.png}
  % \end{figure}
  
  % \begin{center}\chuhao\bfseries\sffamily
  %     博士学位论文
  % \end{center}
  % ~\\[2.5cm]
  % \begin{longtable}{>{\centering}m{15.38cm}}\centering
  %   \erhao\sffamily 论文题目 \tabularnewline \hline
  %   \erhao\sffamily ~ \tabularnewline
  %   \erhao\sffamily ~ \tabularnewline \hline
  % \end{longtable}
  % ~\\[3cm]

  % \begin{longtable}{>{\centering}m{1.7in}>{\centering}m{1.8in}} \centering
  %   \makebox[6em][s]{作者姓名} &  \tabularnewline \cmidrule(lr){2-2} 
  %   \makebox[6em][s]{学科专业} &  \tabularnewline \cmidrule(lr){2-2} 
  %   \makebox[6em][s]{指导教师} &  \tabularnewline \cmidrule(lr){2-2}
  %   \makebox[6em][s]{所在学院} &  \tabularnewline \cmidrule(lr){2-2}
  %   \makebox[6em][s]{论文提交日期} &  \tabularnewline \cmidrule(lr){2-2} 
  % \end{longtable}
  
  
  % \clearpage
  
  % \noindent \textbf{Application of the Wavelet Analysis in}
  
  % \noindent \textbf{the Fault Diagnosis of Rotating Machines}
  % A Dissertation Submitted for the Degree of Doctor of Philosophy
  % \textbf{Candidate:Li Xiaoming}
  
  % \textbf{Supervisor:Prof. Chen Ping}
  
  
  % South China University of Technology 
  
  % Guangzhou, China
  
  % \clearpage
  
  % \noindent \textbf{分类号:                                        学校代号:10561}
  
  % \noindent \textbf{学 号:                                         秘密}$\mathrm{\bigstar}$\textbf{   5年}
  
  % \noindent 华南理工大学博士学位论文
  
  % \noindent \textbf{(论文题名和副题名)}
  
  % \noindent 作者姓名:                       指导教师姓名、职称: 
  
  % \noindent 申请学位级别:                   学科专业名称:
  
  % \noindent 研究方向:
  
  % \noindent 论文提交日期:     年   月   日      论文答辩日期:     年   月   日
  
  % \noindent 学位授予单位:华南理工大学           学位授予日期:     年   月   日
  
  % \noindent 答辩委员会成员:
  
  % \noindent 主席:\underbar{             } 
  
  % \noindent 委员:\underbar{                                                               }
  
  % \noindent 
  
  % \clearpage
  
  % \noindent \textbf{华南理工大学}
  
  % \noindent \textbf{学位论文原创性声明}
  
  % \noindent \textbf{}
  
  % 本人郑重声明:所呈交的论文是本人在导师的指导下独立进行研究所取得的研究成果。除了文中特别加以标注引用的内容外,本论文不包含任何其他个人或集体已经发表或撰写的成果作品。对本文的研究做出重要贡献的个人和集体,均已在文中以明确方式标明。本人完全意识到本声明的法律后果由本人承担。
  
  % 作者签名:                日期:    年   月   日
  
  % \noindent 
  
  % \noindent \textbf{学位论文版权使用授权书}
  
  % \noindent 
  
  % 本学位论文作者完全了解学校有关保留、使用学位论文的规定,即:研究生在校攻读学位期间论文工作的知识产权单位属华南理工大学。学校有权保存并向国家有关部门或机构送交论文的复印件和电子版,允许学位论文被查阅(除在保密期内的保密论文外);学校可以公布学位论文的全部或部分内容,可以允许采用影印、缩印或其它复制手段保存、汇编学位论文。本人电子文档的内容和纸质论文的内容相一致。
  
  % 本学位论文属于:
  
  % □保密(校保密委员会审定为涉密学位论文时间:\underbar{   }年\underbar{  }月\underbar{  }日),于\underbar{   }年\underbar{  }月\underbar{  }日解密后适用本授权书。
  
  % □不保密,同意在校园网上发布,供校内师生和与学校有共享协议的单位浏览;同意将本人学位论文编入有关数据库进行检索,传播学位论文的全部或部分内容。
  
  % (请在以上相应方框内打``$\mathrm{\sqrt{}}$'')
  
  
  
  % 作者签名:                        日期:
  
  % 指导教师签名:                    日期  
  
  % 作者联系电话:                    电子邮箱:
  
  % 联系地址(含邮编):
  
  

\end{committee}



% 也可以导入 Word 版转的 PDF 文件
% \begin{committee}[file=figures/committee.pdf]
% \end{committee}
