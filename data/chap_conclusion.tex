% !TeX root = ../main.tex

\chapter{结{\quad}论}


\section*{研究工作总结}

结论是对论文主要研究结果、论点的提炼与概括,应精炼、准确、完整,使读者看后能全面了解论文的意义、目的和工作内容。
结论是最终的、总体的结论,不是正文各章小结的简单重复。
结论应包括论文的核心观点,主要阐述作者的创造性工作及所取得的研究成果在本领域中的地位、作用和意义,交代研究工作的局限,提出未来工作的意见或建议。
同时,要严格区分自己取得的成果与指导教师及他人的学术成果。
在评价自己的研究工作成果时,要实事求是,除非有足够的证据表明自己的研究是“首次”、“领先”、“填补空白”的,否则应避免使用这些或类似词语。

\section*{研究工作展望}

学位论文的结论单独作为一章排写,但不加章号。
结论是对整个论文主要成果的总结。在结论中应明确指出本研究内容的创造性成果或创新性理论(含新见解、新观点),对其应用前景和社会、经济价值等加以预测和评价,并指出今后进一步在本研究方向进行研究工作的展望与设想。
如果不能导出应有的结论,也可以没有结论而进行必要的讨论。
